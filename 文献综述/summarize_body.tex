\section{前言}

	人类的生产和生活都离不开水,然而随着经济的发展,水资源短缺和水污染问题愈来愈严重。水质监测从而成为保护水资源的预防措施。 

	水质分析仪是用于对水的特性和有害物质进行分析,从而确定水体受污染状况的仪器。它的出现,取代了传统的人工分析,大大提高了水质监测的有效性。在市场需求的推动作用下,水质分析仪的在线化分析已经得到普及,并朝着更高级的智能化和专家化发展。\cite{zaixian}

	在环境样品的毒性检验方面,生物法是必需手段之一。而传统的采用浮游生物、藻类和鱼类等方法,耗时耗费, 不宜用作常规检验。发光菌毒性检验方法是近年来倾向于发展一些快速、简便、经济的检验方法之一。它独特的生理特性, 加之与现代光电检测手段完美匹配的特点而使其倍受关注。

    嵌入式系统是以应用为中心,以计算机技术为基础,软硬件可裁剪,适用于应用系统,对功能、可靠性、成本、体积、功耗等方面有特殊要求的专用计算机系统。\cite{qianrushi}
    
	嵌入式系统一般包含嵌入式微处理器、外围硬件设备、嵌入式操作系统和应用程序4个部分。嵌入式领域已经有丰富的软硬件资源可以选择,涵盖了通信、网络、工业控制、消费电子、汽车电子等各种行业。据调查,目前全世界的嵌入式操作系统已经有两百多种。从20 世纪80 年代开始,出现了一些商用嵌入式操作系统,它们大部分都是为专有系统而开发的。随着嵌入式领域的发展,各种各样嵌入式操作系统相继问世。有许多商业的嵌入式操作系统,也有大量开放源码的嵌入式操作系统。其中著名的嵌入式操作系统有:mC/OS、VxWorks、Neculeus、Linux 和Windows
CE 等。采用嵌入式系统和匹配的传感器能够让仪器更加智能和专业。

    本文拟从水质毒性监测必要性、发光细菌法以及嵌入式操作系统等逐一综述。
%{正文}
\section{水质毒性监测}
\subsection{必要性}
	随着近代工业的发展,人类赖以生存的水生生态系统受到了越来越严重的污染,而且突发性环境污染事故时有发生,如人为投毒、自然灾害引起的水质突变,尤其是石油化工原料、不达标污水的排放以及有毒有害危险品的生产、储存和运输过程中发生的事故所造成的水体污染等。
	
	就我国而言,近年来接连发生多起重大突发性水污染事件,例如:2006.9,湖南新墙河一化工厂污水泄漏导致的致癌物质砷化物进入河流,污染了当地10万居民的饮用水。2010.7,大连石油泄漏,不仅影响了当地渔业、旅游业和附近居民的生活,还对海水质量、海洋生物造成了威胁。可见,水质污染事件不但造成巨大的直接经济损失,而且还要耗费相当大的投资来整治和恢复生态环境。并且,水质污染事件具有发生、发展、危害的不确定性、流域性、影响的长期性和应急主体不明确等特点。这势必要求我们快速地应对各种突发性水体污染事故,尽量减少经济损失和社会影响。
	
	此外,目前对废(污)水排放的监督和管理主要采用理化监测方法进行毒性评价和总量控制,但是此方法具有较大的局限性。因为理化分析手段尽管能够定量地分析物质毒性,但只能得到单一污染物产生的毒性,而受污水体中往往存在多种复杂物质,是一个混合体系,其毒性往往是所有组分污染物拮抗、叠加、协同或抑制的综合结果。即使混合体系中单一组分处于无毒效应时,该组分对混合体系的总毒性仍有一定贡献。\cite{thesingle}为此美国环保局(EPA)推荐了一个综合方案,即通过化学定义的理化标准和全水样毒性限来控制废(污)水的排放质量,后者包括对生物的急性和慢性毒性测试。\cite{waterpolicy}
	
	因此,发展快速、准确、有效的水质毒性监测方法显得非常迫切和必要。	
\subsection{生物检测}
	环境中有毒物质生物毒性的测定与评价,一般用浮游生物、藻类和鱼类等水生生物,以其形态、运动性、生理代谢的变化或者死亡率做指标来评价环境污染物的毒性。这些方法一度成为评价环境污染的必须手段之一,但这些方法操作都比较繁琐,检测时间较长,检测费用较高,且结果不稳定,重复性差,使其难以推广应用,且不适于常规的检验,尤其是现场应急监测。\cite{faguang}针对传统生物毒性检测方法的不足,一些快速、简便、经济、有效的现代检测方法逐步发展起来。其中发光细菌法因其独特的生理特性,与现代光电技术完美匹配的特点而备受关注。
\subsection{发光细菌法}
\subsubsection{原理}
	发光细菌含有荧光素、荧光酶、ATP等发光要素,在有氧条件下通过细胞内生化反应而产生微弱荧光。当细胞活性升高,处于积极分裂状态时,其ATP含量高,发光强度增强。发光细菌在毒物作用下,细胞活性下降,ATP含量水平下降,导致发光细菌发光强度的降低。试验显示,毒物浓度与菌体发光强度呈线性负相关,因而,可以根据发光细菌发光强度判断毒物毒性大小,用发光度表征毒物所在环境的急性毒性。\cite{yingyong}
\subsubsection{发展现状}	
	目前,发光细菌法已经成为了一种简单快速的生物毒性检测手段,广泛应用于质检、环境检测、水产养殖等领域,并被列入了国际标准ISO11348,我国国家标准GB/T15441-1995,德国国家标准DIN38412。在我国,目前主要是以国际ISO标准和我国国标作为依据,尤其是国际ISO标准。
	
	ISO11348标准的检测原理是,在15$^{\circ}C$ 温度下,以无毒参比溶液做对比,样品或其稀释液与Vibrio Fischeri(费氏弧菌)接触15min或30min或5min后,测量出实际样品对发光细菌的抑制率。水质的毒性水平一LID值(当抑制率降低到20\%时样品的稀释倍数)、EC20或EC50值(造成20\%或50\%抑制率时样品的浓度)表示。LID值越高,EC值越低,表明样品的毒性越强。而我国的国家标准GB/T15441-1995采用的菌种是明亮发光杆菌T3小种,水质的毒性水平是以相当的氯化汞浓度或选用EC50值来表征。
	
	目前,国内外常见的快速生物毒性分析仪大多是建立在细菌发光法的原理上,一般有适用于实验室检测的台式毒性仪和适用于现场使用的便携式毒性仪器。前者主要应用于实验室中,而后者主要应用于现场的常规快速监测或突发事件的应急监测。
\subsection{嵌入式}
	水质分析仪的ARM处理器负责接收并处理采集到的各种传感器信号,调用各种算法,把处理后的数据在显示屏上进行显示和绘制曲线,并对步进电机和电磁阀等进行控制。可用的操作系统有很多,以下是对主流操作系统的简介。\cite{os}
\begin{CJKenumerate}
\item VxWorks \\
	\parindent 2em \indent VxWorks操作系统是美国WindRiver公司于1983年设计开发的一种嵌入式实时操作系统(RTOS),是Tornado嵌入式开发环境的关键组成部分。良好的持续发展能力、高性能的内核以及友好的用户开发环境,在嵌人式实时操作系统领域逐渐占据一席之地。VxWorks具有可裁剪微内核结构;高效的任务管理;灵活的任务间通讯;微秒级的中断处理;支持POSIX 1003.1b实时扩展标准;支持多种物理介质及标准的、完整的TCP/IP网络协议等。

	然而其价格昂贵。由于操作系统本身以及开发环境都是专有的,价格一般都比较高,通常需花费10万元人民币以上才能建起一个可用的开发环境,对每一个应用一般还要另外收取版税。一般不通供源代码,只提供二进制代码。由于它们都是专用操作系统,需要专门的技术人员掌握开发技术和维护,所以软件的开发和维护成本都非常高。支持的硬件数量也很有限。
\item Windows CE\\
	\indent Windows CE与Windows系列有较好的兼容性,无疑是Windows CE推广的一大优势。其中WinCE3.0是一种针对小容量、移动式、智能化、32位、了解设备的模块化实时嵌人式操作系统。为建立针对掌上设备、无线设备的动态应用程序和服务提供了一种功能丰富的操作系统平台,它能在多种处理器体系结构上运行,并且通常适用于那些对内存占用空间具有一定限制的设备。它是从整体上为有限资源的平台设计的多线程、完整优先权、多任务的操作系统。它的模块化设计允许它对从掌上电脑到专用的工业控制器的用户电子设备进行定制。操作系统的基本内核需要至少200KB的ROM。由于嵌入式产品的体积、成本等方面有较严格的要求,所以处理器部分占用空间应尽可能的小。系统的可用内存和外存数量也要受限制,而嵌入式操作系统就运行在有限的内存(一般在ROM或快闪存储器)中,因此就对操作系统的规模、效率等提出了较高的要求。从技术角度上讲,Windows CE作为嵌入式操作系统有很多的缺陷:没有开放源代码,使应用开发人员很难实现产品的定制;在效率、功耗方面的表现并不出色,而且和Windows一样占用过的系统内存,运用程序庞大;版权许可费也是厂商不得不考虑的因素。
\item 嵌入式Linux\\
	\indent 这是嵌入式操作系统的一个新成员,其最大的特点是源代码公开并且遵循GPL协议,在近一年多以来成为研究热点,据IDG预测嵌入式Linux将占未来两年的嵌入式操作系统份额的50%。

	由于其源代码公开,人们可以任意修改,以满足自己的应用,并且查错也很容易。遵从GPL,无须为每例应用交纳许可证费。有大量的应用软件可用,其中大部分都遵从GPL,是开放源代码和免费的,可以稍加修改后应用于用户自己的系统。 有大量的免费的优秀的开发工具,且都遵从GPL,是开放源代码的。有庞大的开发人员群体,无需专门的人才,只要懂Unix/Linux和C语言即可。随着Linux在中国的普及,这类人才越来越多。所以软件的开发和维护成本很低。优秀的网络功能,这在Internet时代尤其重要。稳定——这是Linux本身具备的一个很大优点。内核精悍,运行所需资源少,十分适合嵌入式应用。

	支持的硬件数量庞大。嵌入式Linux和普通Linux并无本质区别,PC上用到的硬件嵌入式Linux几乎都支持。而且各种硬件的驱动程序源代码都可以得到,为用户编写自己专有硬件的驱动程序带来很大方便。
\item µC/OS一Ⅱ\\
	\indent µC/OS一Ⅱ是著名的源代码公开的实时内核,是专为嵌入式应用设计的,可用于8位,16位和32位单片机或数字信号处理器(DSP)。它是在原版本µC/OS的基础上做了重大改进与升级,并有了近十年的使用实践,有许多成功应用该实时内核的实例。它的主要特点如下:
\begin{enumerate}
\item 公开源代码,容易就能把操作系统移植到各个不同的硬件平台上。
\item 可移植性,绝大部分源代码是用C语言写的,便于移植到其他微处理器上。
\item 可固化。
\item 可裁剪性,有选择的使用需要的系统服务,以减少斗所需的存储空间。
\item 占先式,完全是占先式的实时内核,即总是运行就绪条件下优先级最高的任务。
\item 多任务,可管理64个任务,任务的优先级必须是不同的,不支持时间片轮转调度法。
\item 可确定性,函数调用与服务的执行时间具有其可确定性,不依赖于任务的多少。
\item  实用性和可靠性,成功应用该实时内核的实例,是其实用性和可靠性的最好证据。
\end{enumerate}
	由于µC/OS一Ⅱ仅是一个实时内核,这就意味着它不像其他实时存在系统那样提供给用户的只是一些API函数接口,还有很多工作需要用户自己去完成。
\end{CJKenumerate}
	根据嵌入式操作系统的选择原则结合上述信息,我们可以对目前主流的操作系统进行比较,结果如表1所示。
\begin{center}
 表1 嵌入式操作系统比较\\[+1.5em]
\begin{tabular}{|l|c|c|c|c|}
\hline  \backslashbox{原则}{操作系统}	&VxWorks 	&WinCE  &µC/OS-Ⅱ 	&Linux \\ 
\hline                     可移植性 	& 差    		& 一般 	&一般  		&好  \\ 
\hline  				可利用资源	&少  		&多  	&一般  		&多  \\ 
\hline  				可定制性		& 差 		&差  	&好  		&好  \\ 
\hline  				成本			& 高 		& 较高 	&低  		&低  \\ 
\hline 
\end{tabular} 
\end{center}
可以看出,针对水质分析仪设备设计,嵌入式Linux是优选。
\section{小结}
	发展快速、准确、有效的水质毒性监测方法非常迫切和必要。而根据发光细菌法检测原理基于嵌入式Linux的水质分析仪不失为一种优选的解决方案。