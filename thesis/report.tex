\documentclass[12pt,a4paper]{article}
\usepackage[ urlcolor = blue, colorlinks = true, citecolor = black, linkcolor = black]{hyperref}
\usepackage[BoldFont,SlantFont]{xeCJK}
\usepackage{graphicx}
%\usepackage{xltxtra}
\usepackage{fancyhdr}
\usepackage{booktabs}
\usepackage{makecell}
\usepackage{fancybox}
\usepackage{bibentry}
\usepackage{natbib}
\usepackage{longtable}
\usepackage{array}
\usepackage{color}

\XeTeXlinebreaklocale{en-us}
\setmainfont{Times New Roman}
\setCJKmainfont{宋体}
\setCJKfamilyfont{song}{宋体}

\punctstyle{CCT}


\makeatletter

\renewcommand{\contentsname}{开题报告目录(全文附后)}
%\renewcommand{\sectionsname}{开题报告目录(全文附后)}

\renewenvironment{thebibliography}[1]{%
      \list{\@biblabel{\@arabic\c@enumiv}}%
           {\settowidth\labelwidth{\@biblabel{#1}}%
            \leftmargin\labelwidth
            \advance\leftmargin\labelsep
            \@openbib@code
            \usecounter{enumiv}%
            \let\p@enumiv\@empty
            \renewcommand\theenumiv{\@arabic\c@enumiv}}%
      \sloppy
      \clubpenalty4000
      \@clubpenalty \clubpenalty
      \widowpenalty4000%
      \sfcode`\.\@m}
     {\def\@noitemerr
       {\@latex@warning{Empty `thebibliography' environment}}%
      \endlist}
\makeatother

%\usepackage[style=plaintop]{floatrow}

\setlength{\parindent}{24pt}
 \setlength{\parskip}{3pt plus1pt minus2pt}
 \setlength{\baselineskip}{20pt plus2pt minus1pt}
% \setlength{\textheight}{21.5true cm}
 \setlength{\textwidth}{15true cm}
  \setlength{\headsep}{10truemm}
  \setlength{\oddsidemargin}{0.26cm}   % 左边 3.25cm=0.71+2.54
\setlength{\evensidemargin}{0.26cm}
    


%\usepackage{tableau}

\title{浙江理工大学信电学院通信工程专业本科毕业论文开题报告}
\author{microcai}

\renewcommand{\arraystretch}{1.8}
 
\begin{document}

{
\fontsize{15}{15.00}
\selectfont{}\textbf{浙江理工大学信电学院通信工程专业本科毕业论文开题报告}
}

\begin{table}[here]
 \begin{tabular}{|m{55pt}|m{180pt}|m{55pt}|m{80pt}|}
  	  \hline
     \makecell{\textbf{班级}} & \makecell{\textbf{07通信(1)班}} & \makecell{\textbf{姓名}} &  				\makecell{\textbf{董璐琦}}  \\
     \hline
     \textbf{课题名称} & \multicolumn{3}{c|}{ \textbf{基于嵌入式技术的水质分析仪控制器关键技术研究}	} \\
     \hline 

	\multicolumn{4}{|m{\textwidth-14pt}|}{
		\tableofcontents
	} \\       
    \hline 
    
    \textbf{成绩} & \multicolumn{3}{l|}{} \\
    \hline
    \makecell{\textbf{答辩}\\\textbf{意见} } & \makecell{答辩组长签名:\\\\   年\space 月 日} & 
	     \makecell{ 
	     	\\\\\textbf{系}\\\textbf{主}\\\textbf{任}\\
	     	\textbf{审}\\\textbf{核}\\ \textbf{意}\\\textbf{见}\\\\
	      } & 	\multicolumn{1}{b{30pt}|}{ 签名  }\\
     \hline
  \end{tabular}
\end{table} 

{

\linespread{1.5}

\section{选题意义与可行性分析}
随着工业进步和社会发展,水污染日趋严重,成了世界性的头号环境治理难题。
水质监测已经成为保护和管理水资源的重要基础,水质监测提供的水质信息尤为重要。
\section{研究的基本内容与拟解决的主要问题}

本文旨在研究从英语生成语法树并对语法树进行汉语表达的可能性。

主要的问题是如何从一段英语文本生成抽象语法树,并使用合理的方式将语法树表达为汉语。

英语具有非常明显的结构,如果能将一个句子的各个单词的词性全部划分,那就非常容易构建语法树。
而为单词标记词性,已经是一个前人研究的结果了。

%TODO 查一下我最近看的到底是哪篇文章讲解了这一过程,加入文献引用库

\section{总体研究思路及预期研究成果}

句子不是词语的堆砌。一个句子是有语法在里面的。 
如果完整的解析一个句子的意思,生成另一个中间语言。这个中间语言被叫做元语言。 在内存中可以用一个抽象语法树完整表达。
这个元语言必须是图灵完全而且是自描述的。 

然后再依据元语言,构造符合目标语言语法的,语义一致的一系列语句。 
接着使用统计模型,从中选择一条最有可能,正确的概率最大的作为最终的翻译结果。 

所以,机器翻译就被分解成几个子任务了。 

首先,需要构建一种元语言。这个语言应该是图灵完全和自描述的,并且可以在内存中用数据结构表示出来。由于我对英语最熟悉,所以我打算使用英语做为基础来构建元语言。 

第二,源语言到元语言的翻译。这个过程就是词法分析。找出每个单词的含义,并结合上下文和所处的环境,唯一确定下对应的元语言。如果含义无法唯一确定,可以构造多条元语言语句。待后续分析后再做决定 

第三,元语言的到目标语言的表达。这个时候必须依靠统计学,因为无法唯一确定元语言单词到目标语言单词的映射。需要基于统计学进行筛选。 

预期做出一个Demo完成翻译任务。

\section{研究工作计划}

年底前拜读机器翻译大师的作品,找到一些前人已经有的功能,避免重复发明轮子。
次年开始构筑翻译Demo
次年3月完成Demo和论文。

}

\section{参考文献}

\bibliography{../thesisbib}
\bibliographystyle{plain}

\end{document}
